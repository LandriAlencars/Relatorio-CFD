\documentclass[12pt, a4paper]{article}

% --- PACOTES ESSENCIAIS ---
\usepackage[utf8]{inputenc}
\usepackage[T1]{fontenc}
\usepackage[brazil]{babel}
\usepackage{graphicx} % Para inserir imagens
\usepackage{amsmath}  % Para equações matemáticas
\usepackage{amssymb}  % Para símbolos matemáticos
\usepackage{geometry} % Para configurar as margens
\usepackage{hyperref} % Para criar links no PDF
\usepackage{booktabs} % Para tabelas com visual profissional
\usepackage{float}    % Para melhor controle de posicionamento de figuras e tabelas

% --- CONFIGURAÇÃO DAS MARGENS ---
\geometry{
    a4paper,
    left=3cm,
    right=2cm,
    top=3cm,
    bottom=2cm
}

% --- CONFIGURAÇÃO DOS LINKS ---
\hypersetup{
    colorlinks=true,
    linkcolor=blue,
    filecolor=magenta,
    urlcolor=cyan,
}

% --- DADOS DO TÍTULO ---
\title{Simulação de Fluxo Laminar em Canal Retangular 3D: \\ Um Estudo de Convergência de Malha}
\author{Seu Nome Completo}
\date{\today}

\begin{document}

\maketitle
\tableofcontents
\newpage

\section{Introdução e Objetivos}

Nesta seção, descreva o contexto do trabalho \cite{bruus}.  
\begin{itemize}
    \item Apresente a importância do estudo de escoamentos em microcanais.
    \item Mencione as aplicações (por exemplo, microfluídica, MEMS, etc.).
    \item Defina claramente os objetivos deste relatório, como comparar resultados numéricos com a solução teórica de Bruus \cite{bruus}, avaliar a influência do refinamento de malha, da vazão e do fluido sobre a solução numérica.
\end{itemize}

\section{Descrição do Problema e Metodologia}

\subsection{Descrição do Problema Físico}

O problema consiste no escoamento laminar, incompressível e estacionário de um fluido em um canal retangular tridimensional.

\paragraph{Geometria:} O canal possui comprimento \(L = 2\ \mathrm{cm}\) e seção transversal quadrada com largura \(W = 100\ \mu\mathrm{m}\) e altura \(H = 100\ \mu\mathrm{m}\).

\paragraph{Condições de Contorno:}
\begin{itemize}
    \item \textbf{Entrada:} Perfil de velocidade uniforme com vazão prescrita.
    \item \textbf{Saída:} Pressão de referência zero.
    \item \textbf{Paredes:} Condição de não-deslizamento (no-slip, \(\vec{u} = 0\)).
\end{itemize}

\subsection{Parâmetros de Estudo}

\paragraph{Fluidos:} As simulações foram realizadas para dois fluidos distintos, cujas propriedades a \(20^\circ\mathrm{C}\) estão listadas na Tabela \ref{tab:fluidos}:

\begin{table}[H]
    \centering
    \caption{Propriedades dos fluidos utilizados.}
    \label{tab:fluidos}
    \begin{tabular}{lcc}
        \toprule
        \textbf{Fluido} & \textbf{Densidade \(\rho\) [kg/m³]} & \textbf{Viscosidade \(\mu\) [Pa·s]} \\
        \midrule
        Água & 1000 & \(1.0 \times 10^{-3}\) \\
        Álcool Isopropílico & 786 & \(2.04 \times 10^{-3}\) \\
        \bottomrule
    \end{tabular}
\end{table}

\paragraph{Vazões e Número de Reynolds:} Foram investigadas quatro vazões volumétricas: \(Q = [1,\ 10,\ 100,\ 1000]\ \mu\mathrm{L/min}\).  
Para cada caso, o número de Reynolds \(Re\) foi calculado para verificar o regime laminar, usando a fórmula
\[
Re = \frac{\rho \, U \, D_h}{\mu},
\]
onde \(U\) é a velocidade média no canal e \(D_h\) é o diâmetro hidráulico.  
Os resultados estão na Tabela \ref{tab:reynolds}.

\begin{table}[H]
    \centering
    \caption{Vazões e números de Reynolds correspondentes para a água.}
    \label{tab:reynolds}
    \begin{tabular}{ccc}
        \toprule
        \textbf{Vazão \(Q\) [\(\mu\)L/min]} & \textbf{Velocidade \(U\) [m/s]} & \textbf{Número de Reynolds \(Re\)} \\
        \midrule
        1 & \dots & \dots \\
        10 & \dots & \dots \\
        100 & \dots & \dots \\
        1000 & \dots & \dots \\
        \bottomrule
    \end{tabular}
    % Repita tabela similar para o Álcool Isopropílico, se desejar.
\end{table}

\paragraph{Malhas:} Foram utilizadas quatro malhas com diferentes refinamentos (Coarse, Normal, Fine, Finer).  
As características de cada malha estão na Tabela \ref{tab:malhas}.

\begin{table}[H]
    \centering
    \caption{Características das malhas utilizadas no estudo.}
    \label{tab:malhas}
    \begin{tabular}{lcc}
        \toprule
        \textbf{Nível} & \textbf{Número de Elementos} & \textbf{Tamanho Médio da Célula [m]} \\
        \midrule
        Coarse & \dots & \dots \\
        Normal & \dots & \dots \\
        Fine & \dots & \dots \\
        Finer & \dots & \dots \\
        \bottomrule
    \end{tabular}
\end{table}

\section{Resultados e Discussão}

\subsection{Perfis de Velocidade}

Nesta seção, apresentamos os perfis de velocidade obtidos numericamente e comparamos com a solução teórica de Bruus.  
O perfil foi extraído na seção central do canal (\(x = L/2\)).

\begin{figure}[H]
    \centering
    % \includegraphics[width=0.8\textwidth]{caminho/para/seu/grafico.png}
    \caption{Comparação entre os perfis de velocidade numéricos (para cada malha) e a solução teórica, para água com vazão de … \(\mu\mathrm{L/min}\).}
    \label{fig:perfil_velocidade}
\end{figure}

Aqui, discuta o que o gráfico mostra: se a solução numérica aproxima-se da teórica conforme a malha é refinada, onde diverge, etc.

\subsection{Análise de Convergência de Malha}

Para quantificar a precisão da solução, calculou-se o erro relativo para cada malha.  
A Figura \ref{fig:convergencia} mostra o comportamento do erro em função do tamanho da malha (gráfico log-log), permitindo estimar a ordem de convergência.

\begin{figure}[H]
    \centering
    % \includegraphics[width=0.7\textwidth]{caminho/para/seu/grafico_convergencia.png}
    \caption{Gráfico log-log do erro relativo em função do tamanho médio da célula.}
    \label{fig:convergencia}
\end{figure}

Discuta: o erro diminui com o refinamento, e qual a ordem de convergência observada (linear, quadrática, etc.).

\subsection{Influência da Vazão e do Fluido}

Aqui, analise como a variação da vazão (e, consequentemente, do número de Reynolds) e a troca do fluido (viscosidade diferente) afetaram os resultados numéricos e a convergência.

\section{Conclusão}

Resuma os principais resultados obtidos.  
Por exemplo:
- Os objetivos foram alcançados?
- Qual foi a precisão da simulação em relação aos parâmetros estudados?
- Quais limitações foram encontradas?
- Sugira trabalhos futuros (por exemplo, estudo de não-linearidades, efeitos de instabilidade, diferentes geometrias, fluxo pulsátil, etc.).

\newpage
\begin{thebibliography}{9}
    \bibitem{bruus}
    Henrik Bruus, \textit{Theoretical Microfluidics}, Oxford University Press, 2008.
    % \bibitem{outra_ref} Outro autor, Título, Editora, Ano.
\end{thebibliography}

\end{document}
